\title{Using The DSpace Requirements Reasoner}
\author{
        Gregory Flanagan \\
}
\date{\today}


\documentclass[12pt]{article}

\usepackage{underscore}


\begin{document}
\maketitle

\section{Parser}
First run the parser over the DsSchema.xml file. This will parse the schema file, and generate a Prolog (.pl) file that will be basis for importing everything in the reasoner. To run the the parser navigate the parser folder in the command line and type:
\begin{verbatim}
    python dspace_parser.py file_to_parse.xml
\end{verbatim}
The parser will generate two Prolog files. One file is \emph{file_to_parse.pl} which has the design representation predicates and rules for the reasoner. You do not need to do anything with this file. The second file is \emph{dspace_inlude.pl}. This file contains all the \emph{glue} to pull the reasoner together.


\section{Starting The Requirements Reasoner}
Navigate to the reasoner's source code root folder (src/). There should be a file named \emph{dspace_include.pl} if you have already run the parser. Fire up SWI-Prolog with the command 
\begin{verbatim} 
   pl
or 
   swipl
\end{verbatim} depending on your OS. Once in Prolog type 
\begin{verbatim} 
  ? [dspace_include].
\end{verbatim} to load the requirements reasoner. This also loads the design file from the parser. Note that only one design can be loaded in the reasoner at a time. There will be some warnings because I am loading several CHR files and I haven't yet found a way to do this cleanly. Now you are ready to use the reasoner.

\section{Using The Requirements Reasoner}
The reasoner is used as a query engine.

\subsection{Positioning}


\subsection{Positioning for Sequences}

\subsection{Facing}

\subsection{Proximity}

\subsection{Visibility}

\subsection{Privacy}

\subsection{Continuity}

\subsection{Enclosure}

\subsection{Artefactual Interference}

\subsection{Spatial Queries}
\bibliographystyle{abbrv}
\bibliography{main}

\end{document}
This is never printed
